\documentclass[pdftex,12pt,a4paper]{article}
\usepackage[pdftex]{graphicx}
\usepackage[utf8]{inputenc}
\usepackage[brazil]{babel}
\usepackage{amsfonts,amssymb,graphicx,enumerate}
\usepackage[centertags]{amsmath}
\usepackage{parskip}
% Configuracoes de pagina
\usepackage[lmargin=3cm,rmargin=3cm,tmargin=3cm,bmargin=3cm]{geometry}

\begin{document}

\title{Projeto 2\\Laboratório de Sistemas Computacionais}
\author{Victor Accarini D'Lima\\RA105753\\Grupo 15}
\date{\today}
\maketitle

\bigskip

\section{Introdução e Objetivos}

Esse projeto tem como objetivo verificar as diferenças de desempenho entre 
diferentes implementações de um processador MIPS utilizando o simulador ArchC. 
Para verificar essas diferenças foram utilizadas 7 configurações diferentes, 
variando entre implementações de um \textit{pipeline} de 5 estágios sem 
\textit{forwards}, sem \textit{branch predictor}, para um pipeline de 5 estágios 
superescalar e um pipeline de 7 estágios ambos com controle de \textit{hazards}, 
para cada uma dessas configurações foram utilizadas também 4 configurações de cache.

Os 3 programas utilizados para a avaliação de desempenho foram: Um algoritmo para 
encontrar o menor caminho entre dois pontos (Dijkstra), um encoder de jpeg (JPEG) e um 
gerador de chaves criptográficas SHA (SHA), todos disponibilizados no repositório do 
professor.

--Escrever sobre os resultados

\section{Metodologia}

Considerando que o simulador disponibilizado executa uma instrução por vez, 
foram implementadas estruturas para simular um pipeline e seus dispositivos 
de controles de \textit{hazards}. Ao rodar os programas de teste no simulador 
os valores das penalidades e ganhos em ciclos de cada configuração simulada 
eram mostrados no final da execução.

Para a simulação dos dispositivos de cache foi utilizado o programa Dinero, 
que dado a configuração de cache pedida e as instruções, gravações e leituras 
da memória retornava a porcentagem de falhas de acesso a cache.

\subsection{Cache}

As quatro configurações de cache selecionadas para a avaliação foram:

\begin{description}
\item[Cache 1] \hfill \\
Cache de instruções:\\
L1 - Tamanho: 2k Blocos: 4bytes 1-Associativo\\
Cache de Dados:\\
L1 - Tamanho: 4k Blocos: 16bytes 2-Associativo
\item[Cache 2] \hfill \\
Cache de instruções:\\
L1 - Tamanho: 1k Blocos: 4bytes 2-Associativo\\
Cache de Dados:\\
L1 - Tamanho: 8k Blocos: 16bytes 4-Associativo
\item[Cache 3] \hfill \\
Cache de instruções:\\
L1 - Tamanho: 2k Blocos: 4bytes 1-Associativo\\
L2 - Tamanho: 8k Blocos: 1bytes 2-Associativo\\
Cache de Dados:\\
L1 - Tamanho: 4k Blocos: 16bytes 4-Associativo\\
L2 - Tamanho: 32k Blocos: 64bytes 4-Associativo
\item[Cache 4] \hfill \\
Cache de instruções:\\
L1 - Tamanho: 1k Blocos: 4bytes 2-Associativo\\
L2 - Tamanho: 8k Blocos: 1bytes 2-Associativo\\
Cache de Dados:\\
L1 - Tamanho: 16k Blocos: 8bytes 1-Associativo\\
L2 - Tamanho: 32k Blocos: 64bytes 4-Associativo
\end{description}

Foram testadas diversas outras configurações para as caches, porém essas configurações foram as que 
retornaram os resultados mais plausíveis, ou seja, com \textit{miss} entre 2-8\% o que é próximo da margem 
encontrada na literatura.

\subsection{Arquiteturas}

Durante a simulação observamos diversos eventos, o primeiro é o nº de instruções executadas, seguido pela 
quantidade de \textit{load-use hazards(LUH)}, \textit{forwards}, \textit{branchs w/ static predictor(BSP)}, 
\textit{branchs w/ local saturating counter predictor(BLSCP)}, \textit{branchs w/ local two-level adaptive 
predictor(BLTAP)} e no caso da arquitetura superescalar o nº de instruções em paralelo.

Algumas ressalvas precisam ser feitas com relação ao BLSCP e BLTAP pois na simulação foi considerado 
que as tabelas utilizadas para guardar os valores referentes a cada branch possui capacidade para 
alocar todos os branchs dos programas utilizados. Na arquitetura superescalar a divisão do pipeline 
ocorre do estágio ID para o EX onde um pipeline trata de instruções de load-store e outro do resto.

A arquitetura com o pipeline de 7 estágios adiciona as funcionalidades do \textit{Translation 
Lookaside Buffer(TLB)} ao processador MIPS colocando 1 estágio de \textit{Instruction 
Translation(IT)} antes do estágio IF e um estágio de \textit{Memory Translation(MT)} antes do 
estágio MEM.

Na tabela abaixo é mostrada cada arquitetura considerada e suas funcionalidades:

\begin{table}[h]
\resizebox{15cm}{!} {
\begin{tabular}{l|lllllll}
Arquiteturas/Funcionalidades  & Modelo 1 & Modelo 2 & Modelo 3 & Modelo 4 & Modelo 5 & Modelo 6 & Modelo 7 \\ \hline
Estágios de \textit{Pipeline} & 5        & 5        & 5        & 5        & 5        & 7        & 7        \\
LUH                           & Não      & Sim      & Sim      & Sim      & Sim      & Sim      & Sim      \\
\textit{Forwards}             & Não      & Sim      & Sim      & Sim      & Sim      & Sim      & Sim      \\
BSP                           & Sim      & Sim      & Não      & Não      & Não      & Não      & Não      \\
BLSCP                         & Não      & Não      & Sim      & Não      & Não      & Sim      & Não      \\
BLTAP                         & Não      & Não      & Não      & Sim      & Sim      & Não      & Sim      \\
Instruções em Paralelo        & Não      & Não      & Não      & Não      & Sim      & Não      & Não      \\ \hline
\end{tabular}
}
\end{table}

\subsection{\textit{Branch Predictors}}

-- Exemplo de como funcionam

\section{Análise de Dados}

Com os três programas selecionados Dijkstra, JPEG e SHA obtivemos os seguintes resultados para os modelos de 1 a 4:

\begin{table}[h]
\resizebox{15cm}{!} {
\begin{tabular}{l|lll|lll|lll|lll|}
                        & Modelo 1  &          &          & Modelo 2  &          &          & Modelo 3  &          &          & Modelo 4  &          &          \\
                        & Dijkstra  & JPEG     & SHA      & Dijkstra  & JPEG     & SHA      & Dijkstra  & JPEG     & SHA      & Dijkstra  & JPEG     & SHA      \\ \hline
Nº de Instruções        & 285280151 & 29474823 & 13036287 & 285280151 & 29474823 & 13036287 & 285280151 & 29474823 & 13036287 & 285280151 & 29474823 & 13036287 \\
LUH                     & 0         & 49232    & 0        & 0         & 49232    & 0        & 0         & 49232    & 0        & 0         & 49232    & 0        \\
Forwards                & 289129896 & 20216111 & 7415403  & 0         & 0        & 0        & 0         & 0        & 0        & 0         & 0        & 0        \\
BSP                     & 74772918  & 6473388  & 2389284  & 74772918  & 6473388  & 2389284  & 0         & 0        & 0        & 0         & 0        & 0        \\
BLSCP                   & 0         & 0        & 0        & 0         & 0        & 0        & 990717    & 772785   & 104520   & 0         & 0        & 0        \\
BLTAP                   & 0         & 0        & 0        & 0         & 0        & 0        & 0         & 0        & 0        & 882597    & 750516   & 119184   \\
Instruções em Paralelo  & 0         & 0        & 0        & 0         & 0        & 0        & 0         & 0        & 0        & 0         & 0        & 0        \\
Cache miss L1-Instrução &           &          &          &           &          &          &           &          &          &           &          &          \\
Cache miss L1-dados     &           &          &          &           &          &          &           &          &          &           &          &          \\
Cache miss L2-Instrução &           &          &          &           &          &          &           &          &          &           &          &          \\
Cache miss L2-dados     &           &          &          &           &          &          &           &          &          &           &          &          \\
                        &           &          &          &           &          &          &           &          &          &           &          &          \\ \hline
Total de Ciclos         & 649182965 & 56213554 & 22840974 & 360053069 & 35997443 & 15425571 & 286270868 & 30296840 & 13140807 & 286162748 & 30274571 & 13155471 \\ \hline
Tempo(s)                &           &          &          &           &          &          &           &          &          &           &          &        \\ \hline 
\end{tabular}
}
\end{table}

\section{Conclusão}

\end{document}

