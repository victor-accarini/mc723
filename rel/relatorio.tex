\documentclass[pdftex,12pt,a4paper]{article}
\usepackage[pdftex]{graphicx}
\usepackage[utf8]{inputenc}
\usepackage[brazil]{babel}
\usepackage{amsfonts,amssymb,graphicx,enumerate}
\usepackage[centertags]{amsmath}
\usepackage{parskip}
% Configuracoes de pagina
\usepackage[lmargin=3cm,rmargin=3cm,tmargin=3cm,bmargin=3cm]{geometry}

\begin{document}

\title{Projeto 2\\Laboratório de Sistemas Computacionais}
\author{Victor Accarini D'Lima\\RA105753\\Grupo 15}
\date{\today}
\maketitle

\bigskip

\section{Introdução e Objetivos}
Esse projeto tem como objetivo verificar as diferenças de desempenho entre 
diferentes implementações de um processador MIPS utilizando o simulador ArchC. 
Para verificar essas diferenças foram utilizadas 7 configurações diferentes, 
variando entre implementações de um \textit{pipeline} de 5 estágios sem 
\textit{forwards}, sem \textit{branch predictor}, para um pipeline de 5 estágios 
superescalar e um pipeline de 7 estágios ambos com controle de \textit{hazards}, 
para cada uma dessas configurações foram utilizadas também 4 configurações de cache.

Os 3 programas utilizados para a avaliação de desempenho foram: Um algoritmo para 
encontrar o menor caminho entre dois pontos (Dijkstra), um encoder de jpeg e um 
gerador de chaves criptográficas SHA, todos disponibilizados no repositório do 
professor.

--Escrever sobre os resultados

\section{Metodologia}
Considerando que o simulador disponibilizado executa uma instrução por vez, 
foram implementadas estruturas para simular um pipeline e seus dispositivos 
de controles de \textit{hazards}. Ao rodar os programas de teste no simulador 
os valores das penalidades e ganhos em ciclos de cada configuração simulada 
eram mostrados no final da execução.

Para a simulação dos dispositivos de cache foi utilizado o programa Dinero, 
que dado a configuração de cache pedida e as instruções, gravações e leituras 
da memória retornava a porcentagem de falhas de acesso a cache.

\subsection{Arquiteturas}

As quatro configurações de cache selecionadas para a avaliação foram:

\begin{description}
\item[Cache 1] \hfill \\
Cache de instruções:\\
L1 - Tamanho: 2k Blocos: 4bytes 1-Associativo\\
Cache de Dados:\\
L1 - Tamanho: 4k Blocos: 16bytes 2-Associativo
\item[Cache 2] \hfill \\
Cache de instruções:\\
L1 - Tamanho: 1k Blocos: 4bytes 2-Associativo\\
Cache de Dados:\\
L1 - Tamanho: 8k Blocos: 16bytes 4-Associativo
\item[Cache 3] \hfill \\
Cache de instruções:\\
L1 - Tamanho: 2k Blocos: 4bytes 1-Associativo\\
L2 - Tamanho: 8k Blocos: 1bytes 2-Associativo\\
Cache de Dados:\\
L1 - Tamanho: 4k Blocos: 16bytes 4-Associativo\\
L2 - Tamanho: 32k Blocos: 64bytes 4-Associativo
\item[Cache 4] \hfill \\
Cache de instruções:\\
L1 - Tamanho: 1k Blocos: 4bytes 2-Associativo\\
L2 - Tamanho: 8k Blocos: 1bytes 2-Associativo\\
Cache de Dados:\\
L1 - Tamanho: 16k Blocos: 8bytes 1-Associativo\\
L2 - Tamanho: 32k Blocos: 64bytes 4-Associativo
\end{description}

\subsection{Benchmarks}

More plain text.

\section{Análise de Dados}

\section{Conclusão}

\end{document}

